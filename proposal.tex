\documentclass[10pt]{article}
\usepackage{geometry}
\usepackage{graphicx}
\usepackage{amsmath, amsthm, amssymb}
\geometry{letterpaper,tmargin=1in,bmargin=1in,lmargin=.75in,rmargin=1in}

 \theoremstyle{plain}
 \newtheorem{theorem}{Theorem}[section]
 \newtheorem{lemma}[theorem]{Lemma}
 \newtheorem{corollary}[theorem]{Corollary}

 \theoremstyle{definition}
 \newtheorem{definition}[theorem]{Definition}

 \theoremstyle{remark}
 \newtheorem{remark}[theorem]{Remark}

\begin{document}
\title{CS 458/535 - Natural Language Processing\\Project Title}
\author{Your Name(s) \\ ID(s)}
\date{}
\maketitle

Use a lot of diagrams to describe the problem and your proposed solution.

\section{Problem Statement}
Clear and concise statement of the problem ... a diagram describing the problem

\section{Motivation}

Why is this problem interesting? Also add a running example in this section.

\section{Background}

What do the readers need to know to understand the problem?

\section{Related Work}
What has already been done?

\noindent Include references to specific (recent) papers, like \cite{awan2021top}.
And other works being done locally \cite{naeem2020subspace}.

\section{Proposed Work}

What exactly do you plan to implement (using a diagram), and how exactly do you plan do it?  

\section{Evaluation Methodology}

How do you plan to evaluate your implementation?

\section{Hypothesis}

What do you hope your experiment/study will show?

\clearpage
\section{Proposed Timeline}
The tentative weekly timeline giving concrete milestones would be as follows. Modify the following timeline for your project.

\begin{itemize}
\item {\bf October 3}: This proposal.
\item
\item {\bf October 10}:Familiarization with ...
\item {\bf October 17}: Implementation of ...
\item {\bf October 30}:
\item {\bf November 7}:
\item {\bf November 14}: Evaluation in progress.
\item
\item {\bf November 21}: Complete experiments.
\item {\bf November 27}: Writing report.
\item {\bf December 5}: Presentation and Final report due.

\end{itemize}


%\nocite{*}
\bibliographystyle{plain}
\bibliography{aim.bib}
% YOU MAY ALSO USE BIBTEX
% \begin{thebibliography}{10}
% 
% \bibitem{lattner:cgo04}
% {Lattner, C., and Adve, V.} 2004.
% \newblock LLVM: A compilation framework for lifelong program analysis \&
% transformation.
% \newblock In {\em Proceedings of the International Symposium on Code generation and Optimization (CGO '04)}, 75--86. 
% \end{thebibliography}

\end{document}
